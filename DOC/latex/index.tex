En aquesta resolució del problema, he definit classes per a representar un \mbox{\hyperlink{class_jugador}{Jugador}}, un \mbox{\hyperlink{class_conjunt_jugadors}{Conjunt\+Jugadors}}, un \mbox{\hyperlink{class_torneig}{Torneig}}, un \mbox{\hyperlink{class_conjunt_tornejos}{Conjunt\+Tornejos}} i una \mbox{\hyperlink{class_classificacio}{Classificacio}}. He pensat que conceptes com una \mbox{\hyperlink{struct_categoria}{Categoria}} i un \mbox{\hyperlink{struct_resultat}{Resultat}} no necessitaven ser classes, ja que no tenien mètodes propis, sino que només emmagatzemaven informació. Per això les he definit com a estructures (struc). A\textquotesingle{}més, en les classes que representen conjunts faig us de diccionaris per a trobar més facilment els objectes. Aquestes classes serveixen per a executar mètodes dels objectes que contenen, principalment. També utilitzo els \mbox{\hyperlink{class_bin_tree}{Bin\+Tree}}, que he agafat d\textquotesingle{}altres practiques anteriors. 